\input phyzzx
\input pstricks
\input pst-plot
\psset{unit=1cm,linewidth=1pt,linecolor=black,arrowscale=2}
%\newrgbcolor{purple}{0.6 0 0.6}
\input hyperbasics

\def\eqdef{\buildrel\rm def\over=}
%\def\bp{{\bf p}}
%\def\bq{{\bf q}}
\def\bk{{\bf k}}
\def\bx{{\bf x}}
\def\by{{\bf y}}
\def\psib{{\smash{\hat{\overline\Psi}}\vphantom{\Psi}}}
%\def\D{\!\not\!}
\def\crossout#1{
    \setbox0=\hbox{$\displaystyle{\strut #1}$}%
    \psline[linecolor=red](0,\ht0)(\wd0,-\dp0)%
    \psline[linecolor=red](0,-\dp0)(\wd0,\ht0)%
    \box0 }
\def\Re{\mathop{\rm Re}\nolimits }
\def\Im{\mathop{\rm Im}\nolimits }
\def\eqdef{\buildrel\rm def\over=}
\def\QED{\hskip 1em $\cal Q.~E.~D.$}
\def\bt{{\bf T}}
\def\comment#1{\langle\!\langle\,\hbox{#1}\,\rangle\!\rangle}

\catcode`\~=12
\def\class{http://bolvan.ph.utexas.edu/~vadim/Classes/2015f}
\catcode`\~=13


\hsize=6.6in	\hoffset=-0.05in
\vsize=8.75in	\voffset=0in
%\overfullrule=0pt

\centerline{\seventeenbf Feynman Propagator of a Scalar Field}
\smallskip
Earlier in class, I have defined the Feynman propagator of a free real scalar field as
a time-ordered correlation function of two scalar fields in the vacuum state,
$$
G_F(x-y)\ \eqdef\ \bra0 \bt \hat\Phi(x)\hat\Phi(y) \ket0 .
\eqn\propdef
$$
We saw that
$$
G_F(x-y)\ =\ \theta(x^0>y^0)\times D(x-y)\ +\ \theta(x^0<y^0)\times D(y-x)\
=\,\cases{ D(x-y) & when $x^0>y^0$,\cr D(y-x) & when $x^0<y^0$,\cr }
\eqn\TwoDs
$$
where
$$
D(x-y)\ \eqdef \int\!{d^3\bk\over(2\pi)^3}\,{1\over 2\omega_\bk}\times\exp\bigl(-ik(x-y)\bigr)^{k^0=+\omega_\bk}.
\eqn\Ddef
$$
A complex scalar field has a similar propagator, but the correlation function  involves one $\hat\Phi$ field
and one $\hat\Phi^\dagger$ field,
$$
\bra0 \bt \hat\Phi^\dagger(x)\hat\Phi(y) \ket0\
=\ \bra0 \bt \hat\Phi(x) \hat\Phi^\dagger(y) \ket 0\
=\ G_F(x-y).\qquad\comment{same $G_F$ as for a real scalar}
\eqno\eq
$$

In these notes, I shall show that the propagator~\propdef\ is a Green's function of the
Klein--Gordon equation, and then I shall explain why there are many different Green's functions
and which particular Green's function happens to be the Feynman propagator.

%%%%%%%%%%%%%%%%%%%%%%%%%%%%%%%%%%%%%%%%%%%%%%%%%%%%%%%%%%%%%%%%%%%%%%%%%%%%%%%%
\bigskip\goodbreak
\leftline{\fourteencp The Feynman propagator is a Green's function}
\smallskip
A free scalar field obeys the Klein--Gordon equation $(\partial^2+m^2)\hat\Phi(x)=0$.
Consequently, the Feynman propagator~\propdef\ for the $\hat\Phi$ is a Green's function
of that equation,
$$
(\partial^2+m^2)G_F(x-y)\ =\ -i\delta^{(4)}(x-y).
\eqn\GreenDef
$$
Note the delta-function on the RHS is in all four dimensions of the spacetime.

To prove eq.~\GreenDef, we start with a
{\bf Lemma: \it the time derivative of a time-ordered product of two operators $\hat A(t)$
and $\hat B(t_0)$ obtains as}
$$
\partder{}{t}\bigl(\bt\hat A(t)\hat B(t_0)\bigr)\
=\ \bt\left(\partder{\hat A(t)}{t}\right)\hat B(t_0)\
+\ \delta(t-t_0)\times\bigl[\hat A(t),\hat B(t_0)\bigr] .
\eqn\Lemma
$$
{\it Proof} (of the lemma):
$$
\eqalignno{
\bt\hat A(t)\hat B(t_0)\
&\eqdef\ \theta(t>t_0)\times\hat A(t)\hat B(t_0)\
	+\ \theta(t<t_0)\times\hat B(t_0)\hat A(t),
&\eq\cr
\partder{}{t}\,\theta(t>t_0)\
&=\ +\delta(t-t_0),\qquad
	\partder{}{t}\,\theta(t<t_0)\ =\ -\delta(t-t_0),
&\eq\cr
}$$
therefore
$$\eqalign{
\partder{}{t}\Bigl(\bt\hat A(t)\hat B(t_0)\Bigr)\
&=\ \partder{}{t}\Bigl(\theta(t>t_0)\times\hat A(t)\hat B(t_0)\Bigr)\
	+\ \partder{}{t}\Bigl(\theta(t<t_0)\times\hat B(t_0)\hat A(t)\Bigr)\cr
&=\ \delta(t-t_0)\times\hat A(t)\times\hat B(t_0)\
	+\ \theta(t>t_0)\times\partder{\hat A(t)}{t}\times\hat B(t_0)\cr
&\qquad-\ \delta(t-t_0)\times\hat B(t_0)\times\hat A(t)\
	+\ \theta(t<t_0)\times\hat B(t)\times\partder{\hat A(t)}{t}\cr
&\qquad\comment{reorganizing terms}\cr
&=\ \delta(t-t_0)\times\Bigl(\hat A(t)\hat B(t_0)\,-\,\hat B(t_0)\hat A(t)\Bigr)\cr
&\qquad+\ \left(
		\theta(t>t_0)\partder{\hat A}{t}\hat B(t_0)\,
		+\,\theta(t<t_0)\hat B(t_0)\partder{\hat A}{t}
		\right)\cr
&=\ \delta(t-t_0)\times\bigl[\hat A(t),\hat B(t_0)\bigr]\
	+\ \bt\left(\partder{\hat A(t)}{t}\times\hat B(t_0)\right).
}\eqno\eq
$$
\QED

Now let's prove that the propagator~\propdef\ is a Green's function.
In light of the lemma~\Lemma,
$$
\eqalign{
\partder{}{x^0}G_F(x-y)\
&=\ \bra0 \partder{}{x^0}\bigl(\bt \hat \Phi(x)\hat\times \Phi(y)\bigr)\ket0\crr
&=\ \bra0 \bt\bigl(\partial_0\hat\Phi(x)\times\hat\Phi(y)\bigr)\ket 0\
	+\ \delta(x^0-y^0)\times\bra0\bigl[\hat\Phi(x),\hat\Phi(y)\bigr]\ket0.\cr
}\eqno\eq
$$
In the second term on the bottom line here, the quantum fields $\hat\Phi(x)$ and $\hat\Phi(y)$
are at equal times $x^0=y^0$, so they commute with each other.
Consequently, the second term vanishes, and we are left with
$$
\partder{}{x^0}G_F(x-y)\
=\ \bra0 \bt\bigl(\partial_0\hat\Phi(x)\times\hat\Phi(y)\bigr)\ket 0.
\eqno\eq
$$
Now let's take another time derivative.
Again, using the lemma~\Lemma, we obtain
$$
\eqalign{
\partial_0^2 G_F(x-y)\ &
=\ \partder{}{x^0}\,\bra0 \bt\bigl(\partial_0\hat\Phi(x)\times\hat\Phi(y)\bigr)\ket 0\crr
&=\ \bra0 \bt\bigl(\partial_0^2\hat\Phi(x)\times\hat\Phi(y)\bigr)\ket 0\
	+\ \delta(x^0-y^0)\times\bra0\bigl[\partial_0\hat\Phi(x),\hat\Phi(y)\bigr]\ket 0.
}\eqn\SecondTD
$$
This time, in the second term on the bottom line, $\partial_0\hat\Phi(x)=\hat\Pi(x)$,
and at equal times $x^0=y^0$, $\hat\Pi(x)$ does {\it not} commute with the $\hat\Phi(y)$.
Instead,
$$
{\rm for}\ x^0=y^0,\quad
\bigl[\hat\Pi(x),\hat\Phi(y)\bigr]\ =\ -i\delta^{(3)}(\bx-\by),
\eqno\eq
$$
hence
$$
\delta(x^0-y^0)\times\bra0\bigl[\partial_0\hat\Phi(x),\hat\Phi(y)\bigr]\ket 0\
=\ -i\delta^{(3)}(\bx-\by)\times\delta(x^0-y^0)\ =\ -i\delta^{(4)}(x-y).
\eqno\eq
$$
Thus, eq.~\SecondTD\ reduces to
$$
\partial_0^2 G_F(x-y)\
=\ \bra0 \bt\bigl(\partial_0^2\hat\Phi(x)\times\hat\Phi(y)\bigr)\ket 0\
-\ i\delta^{(4)}(x-y).
\eqn\TimePart
$$

Now consider the space-derivative terms in the Klein-Gordon equation.
Since the space derivatives commute with the time-ordering, we have
$$
\nabla^2_x G_F(x-y)\
=\ \nabla^2_x\bra0 \bigl(\bt\hat\Phi(x)\times\hat\Phi(y)\bigr)\ket0\
=\ \bra0\ \bt\bigl( \nabla^2\hat\Phi(x)\times\hat\Phi(y)\bigr)\ket0
\eqno\eq
$$
without any extra terms.
Combining this formula with eq.~\TimePart, we obtain
$$
\bigl(\partial_0^2-\nabla^2+m^2\bigr) G_F(x-y)\
=\ \bra0 \bt\bigl( (\partial_0^2-\nabla^2+m^2)\hat\Phi(x)\times\hat\Phi(y)\bigr)\ket0\
-\ i\delta^{(4)}(x-y).
\eqno\eq
$$
On the RHS of this formula, the quantum field $\hat\Phi(x)$ satisfies the Klein--Gordon equation
$(\partial_0^2-\nabla^2+m^2)\hat\Phi(x)=0$, which kills the first term.
Only the second term --- the delta function --- survives on the RHS, thus
$$
\bigl(\partial_0^2-\nabla^2+m^2\bigr) G_F(x-y)\ =\ -i\delta^{(4)}(x-y),
\eqno\eq
$$
which proves that $G_F(x-y)$ is a Green's function of the Klein--Gordon equation.
\QED

%%%%%%%%%%%%%%%%%%%%%%%%%%%%%%%%%%%%%%%%%%%%%%%%%%%%%%%%%%%%%%%%%%%%%%%%%%%%%%%%%%%%
\bigskip\goodbreak
\leftline{\fourteencp General Green's functions and the Feynman's choice}
\smallskip
In general, the same differential equation may have many different Green's functions,
depending on the boundary conditions, \etc{}
So let's consider a generic Green's function of the Klein--Gordon equation, that is,
some function $G(x-y)$ satisfying
$$
(\partial^2+m^2)G(x-y)\ =\ -i\delta^{(4)}(x-y).
\eqn\Greens
$$
Let's Fourier transform this function in all four dimensions,
$$
G(x-y)\ =\int\!{d^4 k\over(2\pi)^4}\,e^{-ik(x-y)}\times\tilde G(k).
\eqno\eq
$$
In the 4--momentum space, eq.~\Greens\ becomes
$$
(-k^2+m^2)\times \tilde G(k)\ =\ -i,
\eqno\eq
$$
hence {\it naively}
$$
G(k)\ =\ {i\over k^2-m^2}
\eqno\eq
$$
and therefore
$$
G(x-y)\ =\int\!{d^4 k\over(2\pi)^4}\,{i\, e^{-ik(x-y)}\over k^2-m^2}\,.
\eqn\NaiveG
$$

The problem with this naive formula is that it integrates over the singularities of the integrand.
Indeed, the denominator $k^2-m^2=k_0^2-\bk^2-m^0$ vanishes on the mass shells $\bk^0=\pm\sqrt{\bk^2+m^2}$,
so we have two 3D families of poles.
In general, an integral of a singular function over its pole is ill-defined, and we must regularize it
to get a definite answer.
For the Green's function in question, we must regulate two 3D-families of poles, thus
$$
G(x)\ =\int_{\rm reg}\!{d^4 k\over(2\pi)^4}\,{i e^{-ik(x-y)}\over k^2-m^2}\
=\int\!{d^3\bk\over(2\pi)^3}\,e^{i\bx\cdot\bk}\times
	\int_{\rm reg}\!{dk_0\over 2\pi}\,{ie^{-itk_0}\over k_0^2-\bk^2-m^2}\,.
\eqn\GREG
$$
In other words, we integrate over the $k^0$ before we integrate over the $\bk$.
In the $\int\!dk^0$ integral, we encounter two simple poles at $k^0=\pm\omega_\bk$,
and we must somehow regularize them to get a definite result.
Only then we integrate that result over $\bk$; hopefully, {\it that} integral
does not encounter any singularities.

Alas, the devil is in the details: There are many different ways to regularize
an integral, and regulators yield different regularized integrals --- which eventually yield
many different Green's functions~\GREG\ of the same Klein--Gordon equation.

In these notes, we are going to use a particularly simple way to regulate an integral
over a simple pole
--- shift the pole away from the real axis into the complex plane,
$$
\int_{\rm reg}\!dx\,{f(x)\over x-x_0}\
=\int\!dx\,{f(x)\over x-(x_0\pm i\epsilon)}
\eqn\Regulator
$$
for an infinitesimal $\epsilon\to+0$.
Equivalently, we may leave the pole real but deform the integration contour slightly away from the real axis
so that it bypasses the pole,
$$
\pspicture(0,-0.5)(13,+0.5)
\psset{linecolor=blue,linewidth=2.5pt,arrowscale=1}
\psline[linestyle=dotted](0,0)(1,0)
\psline[linestyle=dotted](5,0)(6,0)
\psline[linestyle=dotted](7,0)(8,0)
\psline[linestyle=dotted](12,0)(13,0)
\psline(1,0)(2.5,0)
\psarc(3,0){0.5}{0}{180}
\psline{->}(3.5,0)(5,0)
\psline{->}(8,0)(12,0)
\pscircle*[linecolor=red](3,0){0.1}
\pscircle*[linecolor=red](10,-0.5){0.1}
\psset{linecolor=black,linewidth=0.5pt}
\psline(0,0)(6,0)
\psline(7,0)(13,0)
\rput(6.5,0){=}
\rput[l](10.3,-0.5){$x_0-i\epsilon$}
\endpspicture
$$
or
$$
\pspicture(0,-0.5)(13,+0.5)
\psset{linecolor=blue,linewidth=2.5pt,arrowscale=1}
\psline[linestyle=dotted](0,0)(1,0)
\psline[linestyle=dotted](5,0)(6,0)
\psline[linestyle=dotted](7,0)(8,0)
\psline[linestyle=dotted](12,0)(13,0)
\psline(1,0)(2.5,0)
\psarc(3,0){0.5}{180}{360}
\psline{->}(3.5,0)(5,0)
\psline{->}(8,0)(12,0)
\pscircle*[linecolor=red](3,0){0.1}
\pscircle*[linecolor=red](10,+0.5){0.1}
\psset{linecolor=black,linewidth=0.5pt}
\psline(0,0)(6,0)
\psline(7,0)(13,0)
\rput(6.5,0){=}
\rput[l](10.3,+0.5){$x_0+i\epsilon$}
\endpspicture
$$
Note that the contour above the pole and the contour below the pole make for
different regulators and yield different regularized integrals.
Indeed, the two contours differ from each other by a loop around the pole,
which picks up the pole's resudue.
Equivalently, comparing integrals over the real axis over poles shifted up vs. down
in the complex plane, we get results differening by
$$
\int\!dx\,{f(x)\over x-(x_0+i\epsilon)}\ -\int\!dx\,{f(x)\over x-(x_0-i\epsilon)}\
=\ 2\pi i\times f(x_0).
\eqno\eq
$$

In the context of the integral~\GREG, there are two poles in the $\int\! dk^0$ for every $\bk$,
so we must make our choices.
For the sake of Lorentz invariance, we should use the same regulator for every $\bk$, which leaves
with $2\times2=4$ choices:
\item\bullet
Move the pole at $k^0=+\omega_\bk$ to $+\omega_\bk+i\epsilon$ or to $+\omega_\bk-i\epsilon$.
\item\bullet
Move the pole at $k^0=-\omega_\bk$ to $-\omega_\bk+i\epsilon$ or to $-\omega_\bk-i\epsilon$.
\par\goodbreak\noindent
The 4 choices give rise to 4 distinct Lorentz-invariant Green's functions, namely:
\pointbegin %1
{\it Causal retarded Green's function} $G_R$ for poles at $k_0=\pm\omega_k-i\epsilon$,
$$
\pspicture(-6,-0.5)(+6,+0.5)
\psset{linecolor=blue,linewidth=2.5pt,arrowscale=1}
\psline[linestyle=dotted](-6,0)(-5,0)
\psline[linestyle=dotted](+6,0)(+5,0)
\psline{->}(-5,0)(+5,0)
\pscircle*[linecolor=red](-3,-0.5){0.1}
\pscircle*[linecolor=red](+3,-0.5){0.1}
\endpspicture
$$
\point %2
{\it Causal advanced Green's function} $G_A$ for poles at $k_0=\pm\omega_k+i\epsilon$,
$$
\pspicture(-6,-0.5)(+6,+0.5)
\psset{linecolor=blue,linewidth=2.5pt,arrowscale=1}
\psline[linestyle=dotted](-6,0)(-5,0)
\psline[linestyle=dotted](+6,0)(+5,0)
\psline{->}(-5,0)(+5,0)
\pscircle*[linecolor=red](-3,+0.5){0.1}
\pscircle*[linecolor=red](+3,+0.5){0.1}
\endpspicture
$$
\point %3
{\it Time-ordered Green's function} $G_F$ for poles at $k_0=\pm(\omega_k-i\epsilon)$,
$$
\pspicture(-6,-0.5)(+6,+0.5)
\psset{linecolor=blue,linewidth=2.5pt,arrowscale=1}
\psline[linestyle=dotted](-6,0)(-5,0)
\psline[linestyle=dotted](+6,0)(+5,0)
\psline{->}(-5,0)(+5,0)
\pscircle*[linecolor=red](-3,+0.5){0.1}
\pscircle*[linecolor=red](+3,-0.5){0.1}
\endpspicture
$$
This Green's function is the Feynman's propagator~\propdef.
\point %4
{\it Anti-time-ordered Green's function} $G_{AT}$ for poles at $k_0=\pm(\omega_k+i\epsilon)$,
$$
\pspicture(-6,-0.5)(+6,+0.5)
\psset{linecolor=blue,linewidth=2.5pt,arrowscale=1}
\psline[linestyle=dotted](-6,0)(-5,0)
\psline[linestyle=dotted](+6,0)(+5,0)
\psline{->}(-5,0)(+5,0)
\pscircle*[linecolor=red](-3,-0.5){0.1}
\pscircle*[linecolor=red](+3,+0.5){0.1}
\endpspicture
$$

%%%%%%%%%%%%%%%%%%%%%%%%%%%%%%%%%%%%%%%%%%%%%%%%%%%%%%%%%%%%%%%%%%%%%%%%%%%%%%%%%%%%%%
\bigskip\goodbreak
\leftline{\bf Feynman's Choice}
Let's focus on the Feynman's choice of the poles at $+\omega_\bk-i\epsilon$ and $-\omega_\bk+i\epsilon$.
Altogether, the denominator of the integrand in eq.~\GREG\ is
$$
(k_0-\omega_\bk+i\epsilon)\times(k_0+\omega_\bk-i\epsilon)\
=\ k_0^2\,-\,(\omega_\bk-i\epsilon)^2\
\approx\ k_0^2\,-\,\omega_\bk^2\,+\,2i\omega_\bk\epsilon\
=\ k_0^2\,-\,\bk^2\,-\,m^2\,+\,i\epsilon\times2\omega_\bk.
\eqno\eq
$$
In the last expression, we may replace $\epsilon\times2\omega_\bk$ with simply $\epsilon$,
since all we care about is is that it's a positive infinitesimal number${}\to+0$.
Thus
$$
{\rm denominator}\ =\ k_0^2\ -\ \bk^2\ -\ m^2\ +\ i\epsilon\ =\ k^2\ -\ m^2\ +\ i\epsilon,
\eqno\eq
$$
hence manifestly Lorentz invariant expression for the Feynman's Green's function as
$$
G_F(x-y)\ =\int\!{d^4k\over(2\pi)^4}\,{ie^{-ikx}\over k^2-m^2{\red+i\epsilon}}\,.
\eqn\InvarInt
$$

In this section of the notes, we shall see that this Green's function is precisely the
Feynman propagator~\propdef.
Without loss of generality, let's set $y=0$.
In light of eq.~\TwoDs, we expect two different cases according to the sign of the $t=x^0$.
Let's start with the $t>0$ case and deal with $t<0$ later.

We begin to evaluate the 4D  integral~\InvarInt\ by integrating over the $k_0$ for a fixed $\bk$,
$$
\eqalignno{
I(t,\omega_\bk)\ &
=\int\!{dk_0\over 2\pi}\,{ie^{-it k_0}\over k_0^2-\omega_\bk^2+i\epsilon}\,,&
\eqname\Kzero\cr
\noalign{\vskip 10pt}
{\rm then}\quad G_F(\bx,t)\ &
=\int\!{d^3\bk\over(2\pi)^3}\,e^{i\bx\cdot\bk}\times I(t,\omega_\bk) &
\eqname\ThreeK\cr
}$$
In the integral~\Kzero, the integration contour is the real axis, while the two poles lie near the axis
--- but not quite on it --- as on the following diagram
$$
\pspicture(-6,-0.5)(+6,+0.5)
\psset{linecolor=blue,linewidth=2.5pt,arrowscale=1}
\psline[linestyle=dotted](-6,0)(-5,0)
\psline[linestyle=dotted](+6,0)(+5,0)
\psline{->}(-5,0)(+5,0)
\pscircle*[linecolor=red](-3,+0.5){0.1}
\pscircle*[linecolor=red](+3,-0.5){0.1}
\endpspicture
\eqn\Fline
$$
Outside the real axis, the exponential $e^{-itk_0}$ --- with positive $t$ --- rapidly decreases
for large negative $\Im(k_0)$.
Consequently, we may close the integration contour by adding to it a large semicircular arc
in the negative $\Im(k_0)$ half of the complex plane.
Thus,
$$
I(t,\omega_\bk)\ =\ \oint_\Gamma{dk_0\over2\pi}\,
{ie^{-itk_0}\over(k_0-\omega_\bk+i\epsilon)(k_0+\omega_\bk-i\epsilon)}
\eqn\Contourint
$$
where
$$
\Gamma\ =\quad
\pspicture[shift=-5.9](-6,-6)(+6,+1)
\psset{linecolor=blue,linewidth=2.5pt,arrowscale=1}
\psline{->}(-5,0)(+5,0)
\psarc{<-}(0,0){5}{180}{360}
\pscircle*[linecolor=red](-3,+0.5){0.1}
\pscircle*[linecolor=red](+3,-0.5){0.1}
\psset{linecolor=black,linewidth=0.5pt,arrowscale=2}
\psline(-6,0)(+6,0)
\psline(0,-6)(0,+1)
\endpspicture
\eqn\BottomArc
$$
The closed-contour integrals like~\Contourint\ may be evaluated in terms of residues
at the poles {\it surrounded by the contour.}
For the contour~\BottomArc\ at hand, the pole at $+\omega_\bk-i\epsilon$ lies inside
the contour while the other pole lies outside the contour.
Consequently,
$$
I(t,\omega_\bk)\ =\ -2\pi i\times{\rm Residue\ at}\ k_0=+\omega_\bk-i\epsilon ,
\eqno\eq
$$
where the overall $-2\pi i$ factor is due to clockwise direction of the contour.
Specifically,
$$
\eqalign{
I(t,\omega_\bk)\ &
=\ -2\pi i\times\left(
	{ie^{-itk_0}\over2\pi\times\crossout{(k_0-\omega_\bk+i\epsilon)}\times(k_0+\omega_\bk-i\epsilon)}
	\right)_{k_0=+\omega_\bk-i\epsilon}\crr
&=\ +{\exp(-it(\omega_\bk-i\epsilon))\over2(\omega_\bk-i\epsilon)}\cr
&\qquad\comment{taking the $\epsilon\to+0$ limit, which is non-singular}\cr
&=\ +{e^{-it\omega_\bk}\over 2\omega_\bk}\,.\cr
}\eqno\eq
$$
Plugging this result into eq.~\ThreeK, we have
$$
G_F(x)\ =\int\!{d^3\bk\over(2\pi)^3}\,e^{i\bx\cdot\bk}\times{e^{-it\omega_\bk}\over 2\omega_\bk}\
=\int\!{d^3\bk\over(2\pi)^3}\,{1\over2\omega_\bk}\,\exp(i\bx\cdot\bk-it\omega_\bk)\
=\ D(x),
\eqn\PositiveT
$$
in perfect agreement with the Feynman propagator~\propdef\ for $\red t>0$, \cf~eq.~\TwoDs.

Now let's turn to the $t<0$ case.
Again, we need to take the integral
$$
I(t,\omega_\bk)\ 
=\int\!{dk_0\over 2\pi}\,{ie^{-it k_0}\over k_0^2-\omega_\bk^2+i\epsilon}
\eqno\Kzero
$$
along the real axis, bypassing the poles according to
$$
\pspicture(-6,-0.5)(+6,+0.5)
\psset{linecolor=blue,linewidth=2.5pt,arrowscale=1}
\psline[linestyle=dotted](-6,0)(-5,0)
\psline[linestyle=dotted](+6,0)(+5,0)
\psline{->}(-5,0)(+5,0)
\pscircle*[linecolor=red](-3,+0.5){0.1}
\pscircle*[linecolor=red](+3,-0.5){0.1}
\endpspicture
\eqno\Fline
$$
However, for a negative $t$, the exponential $e^{-itk_0}$ decreases for large positive $\Im(k_0)$
(rather than large negative $\Im(k_0)$ as we had for positive $t$), so to close the integration contour
\Fline\ we should add a large semicircular arc in the positive half of the complex plane.
Thus,
$$
I(t,\omega_\bk)\ =\ \oint_{\Gamma'}{dk_0\over2\pi}\,
{ie^{-itk_0}\over(k_0-\omega_\bk+i\epsilon)(k_0+\omega_\bk-i\epsilon)}
\eqn\ContourintII
$$
where
$$
\Gamma'\ =\quad
\pspicture[shift=-0.9](-6,-1)(+6,+6)
\psset{linecolor=blue,linewidth=2.5pt,arrowscale=1}
\psline{->}(-5,0)(+5,0)
\psarc{->}(0,0){5}{0}{180}
\pscircle*[linecolor=red](-3,+0.5){0.1}
\pscircle*[linecolor=red](+3,-0.5){0.1}
\psset{linecolor=black,linewidth=0.5pt,arrowscale=2}
\psline(-6,0)(+6,0)
\psline(0,-1)(0,+6)
\endpspicture
\eqn\TopArc
$$
Unlike the contour~\BottomArc\ which we have used for positive $t$, the contour~\TopArc\
surrounds the negative-frequency pole at $k_0=-\omega_\bk+i\epsilon$.
It is also counterclockwise, hence
$$
\eqalign{
I(t,\omega_\bk)\ &
=\ +2\pi i\times{\rm Residue\ at}\ k_0=-\omega_\bk+i\epsilon\cr
&=\ +2\pi i\times\left(
	{ie^{-itk_0}\over2\pi\times(k_0-\omega_\bk+i\epsilon)\times\crossout{(k_0+\omega_\bk-i\epsilon)}}
	\right)_{k_0=-\omega_\bk+i\epsilon}\crr
&=\ -{\exp(-it(-\omega_\bk+i\epsilon))\over2(-\omega_\bk+i\epsilon)}\cr
&\qquad\comment{taking the $\epsilon\to+0$ limit, which is non-singular}\cr
&=\ +{e^{+it\omega_\bk}\over 2\omega_\bk}\,.\cr
}\eqno\eq
$$
Plugging this $k_0$ integral into the $\int\!d^3\bk$ integral~\ThreeK, we obtain
$$
G_F(\bx,t)\
=\int\!{d^3\bk\over(2\pi)^3}\,e^{+i\bx\cdot\bk}\times{e^{+it\omega_\bk}\over 2\omega_\bk}
=\ D(+\bx,-t).
$$
At first blush, this is not quite the answer we want, but fortunately $D$ is invariant under orthochronous
Lorentz transformation, and in particular under any rotations of the 3D space.
Consequently
$$
D(+\bx,-t)\ =\ D(-\bx,-t),
\eqno\eq
$$
and therefore
$$
{\rm for}\ t<0,\quad G_F(x)\ =\ D(-x),
\eqn\NegativeT
$$
in perfect agreement with eq.~\TwoDs.

Altogether, eqs.~\PositiveT\ and \NegativeT\ tell us that the Feynman's Green's function
$$
G_F(x-y)\ =\int\!{d^4k\over(2\pi)^4}\,{ie^{-ikx}\over k^2-m^2{\red+i\epsilon}}\
=\,\left.\cases{D(x-y) & when $x^0>y^0$\cr D(y-x) & when $x^0<y^0$\cr }\right\}\,
=\ \bra0\bt\hat\Phi(x)\hat\Phi(y)\ket 0
\eqno\eq
$$
is precisely the time-ordered correlation function of two free scalar fields.

%%%%%%%%%%%%%%%%%%%%%%%%%%%%%%%%%%%%%%%%%%%%%%%%%%%%%%%%%%%%%%%%%%%%%%%%%%
\bigskip\goodbreak
\leftline{\bf Other Green's functions}
Besides the Feynman's time-ordered Green's function, there are other useful Green's functions
(of the same Klein-Gordon equation) which obtain for other choices of regularizing the poles.
Of particular interest is the causal retarded Green's function
$$
G_R(x-y)\ =\int\!{d^3\bx\over(2\pi)^3}\,e^{i(\bx-\by)\bk}\times
\int\,{dk_0\over2\pi}\,{i\,e^{-i(x^0-y^0)k_0}\over(k_0-\omega_\bk+i\epsilon)(k_0+\omega_\bk+i\epsilon)}\,,
\eqno\eq
$$
which obtains by shifting both poles below the real axis,
$$
\pspicture(-6,-0.5)(+6,+0.5)
\psset{linecolor=blue,linewidth=2.5pt,arrowscale=1}
\psline[linestyle=dotted](-6,0)(-5,0)
\psline[linestyle=dotted](+6,0)(+5,0)
\psline{->}(-5,0)(+5,0)
\pscircle*[linecolor=red](-3,-0.5){0.1}
\pscircle*[linecolor=red](+3,-0.5){0.1}
\endpspicture
\eqn\Rline
$$
As before, we close this contour by adding a large semicircular arc in the lower or upper half
of the complex plane, depending on the sign of the time difference $t=x^0-y^0$.
In particular, for $t<0$ we close the contour above the real axis,
$$
\Gamma'\ =\quad
\pspicture[shift=-0.9](-6,-1)(+6,+6)
\psset{linecolor=blue,linewidth=2.5pt,arrowscale=1}
\psline{->}(-5,0)(+5,0)
\psarc{->}(0,0){5}{0}{180}
\pscircle*[linecolor=red](-3,-0.5){0.1}
\pscircle*[linecolor=red](+3,-0.5){0.1}
\psset{linecolor=black,linewidth=0.5pt,arrowscale=2}
\psline(-6,0)(+6,0)
\psline(0,-1)(0,+6)
\endpspicture
\eqno\eq
$$
which puts both poles outside the contour.
Consequently, the contour integral vanishes altogether, thus
$$
G_R(x-y)\ =\ 0\quad{\rm when}\ x^0-y^0<0.
\eqno\eq
$$
This is why this Green's function is called {\it retarded:}
time-wise, the point $x$ must follow the point $y$, hence in the context of a source $j(y)$
and the induced field
$$
\phi(x)\ =\int\!d^4y\,G_R(x-y)\times j(y),
\eqno\eq
$$
the source at point $y$ affects the field $\phi(x)$ only at later times $x^0>y^0$ that the field.

Now let's see what $G_R(x-y)$ looks like for $t=x^0-y^0>0$.
This time, we close the contour~\Rline\ below the real axis,
$$
\Gamma\ =\quad
\pspicture[shift=-5.9](-6,-6)(+6,+1)
\psset{linecolor=blue,linewidth=2.5pt,arrowscale=1}
\psline{->}(-5,0)(+5,0)
\psarc{<-}(0,0){5}{180}{360}
\pscircle*[linecolor=red](-3,-0.5){0.1}
\pscircle*[linecolor=red](+3,-0.5){0.1}
\psset{linecolor=black,linewidth=0.5pt,arrowscale=2}
\psline(-6,0)(+6,0)
\psline(0,-6)(0,+1)
\endpspicture
\eqno\eq
$$
so both poles are inside the contour.
Consequently,
$$
\eqalign{
I_R(t,\omega)\ &
=\int_\Gamma\!{dk_0\over2\pi}\,{i\,e^{-itk_0}\over(k_0-\omega+i\epsilon)(k_0+\omega+i\epsilon)}\cr
&=\ -2\pi i\times{\rm Residue~@}(k_0=+\omega-i\epsilon)\
   -\ 2\pi i\times{\rm Residue~@}(k_0=-\omega-i\epsilon)\crr
&=\ {-2\pi i\over2\pi}\times\left(
		{ie^{-itk_0}\over\crossout{(k_0-\omega+i\epsilon)}\times(k_0+\omega+i\epsilon)}
		\right)_{k_0=+\omega-i\epsilon}\cr
&\qquad+\ {-2\pi i\over2\pi}\times\left(
		{ie^{-itk_0}\over(k_0-\omega+i\epsilon)\times\crossout{(k_0+\omega+i\epsilon)}}
		\right)_{k_0=-\omega-i\epsilon}\crr
&=\ +{e^{-it(\omega-i\epsilon)}\over2(\omega-i\epsilon)}\
	+\ {e^{-it(-\omega-i\epsilon)}\over2(-\omega-i\epsilon}\crr
&=\ +{e^{-it\omega}\over 2\omega}\ -\ {e^{+it\omega}\over 2\omega}\,.\cr
}\eqno\eq
$$
Plugging this result into the $\int\!d^\bk$ integral, we obtain
$$
\eqalign{
{\rm For}\ x^0>y^0,\quad
G_R(x-y)\ &
=\int\!{d^3\bk\over(2\pi)^3}\,e^{i\bk(\bx-\by)}\times {e^{-it\omega_\bk}\,-\,e^{+it\omega_\bk}\over2\omega_\bk}\crr
&=\ D(\bx-\by;t)\ -\ D(\bx-\by;-t)\cr
&=\ D(\bx-\by;t)\ -\ D(\by-\bx;-t)\cr
&=\ D(x-y)\ -\ D(y-x).\cr
}\eqno\eq
$$
Note that the bottom line here vanishes for spacelike $(x-y)$, which makes the Green's function $G_R$
not only retarded but also causal: it vanishes unless $y$ lies in the future light cone from $x$.

Similar to the causal retarded Green's function $G_R(x-y)$ we can make the causal advanced Green's function
$G_A(x-y)$ by shifting both poles above the real axis,
$$
\eqalignno{
G_A(x-y)\ &
=\int\!{d^3\bx\over(2\pi)^3}\,e^{i(\bx-\by)\bk}\times
\int\,{dk_0\over2\pi}\,{i\,e^{-i(x^0-y^0)k_0}\over(k_0-\omega_\bk-i\epsilon)(k_0+\omega_\bk-i\epsilon)} &
\eq\cr
&\pspicture(-6,-0.5)(+6,+1.5)
\psset{linecolor=blue,linewidth=2.5pt,arrowscale=1}
\psline[linestyle=dotted](-6,0)(-5,0)
\psline[linestyle=dotted](+6,0)(+5,0)
\psline{->}(-5,0)(+5,0)
\pscircle*[linecolor=red](-3,+0.5){0.1}
\pscircle*[linecolor=red](+3,+0.5){0.1}
\endpspicture &
\eq\cr
}$$
As its name suggests, this Green's function vanishes unless $y$ is in the {\it past} light cone from $x$.

Finally, the fourth choice of regularized poles
$$
\pspicture(-6,-0.5)(+6,+0.5)
\psset{linecolor=blue,linewidth=2.5pt,arrowscale=1}
\psline[linestyle=dotted](-6,0)(-5,0)
\psline[linestyle=dotted](+6,0)(+5,0)
\psline{->}(-5,0)(+5,0)
\pscircle*[linecolor=red](-3,-0.5){0.1}
\pscircle*[linecolor=red](+3,+0.5){0.1}
\endpspicture
\eqno\eq
$$
produce the anti-time-ordered Green's function
$$
G_{AT}(x-y)\ =\int\!{d^4k\over(2\pi)^4}\,{i\,e^{-ik(x-y)}\over k^2-m^2-i\epsilon}\
=\,\cases{ -D(y-x) & when $x^0>y^0$,\cr -D(x-y) & when $y^0>x^0$.\cr }
\eqno\eq
$$

%%%%%%%%%%%%%%%%%%%%%%%%%%%%%%%%%%%%%%%%%%%%%%%%%%%%%%%%%%%%%%%%%%%%%%%%%%%%%%%%%%%%%%
\bigskip\goodbreak
\leftline{\fourteencp Propagators for non-scalar fields}
\smallskip
Let me conclude these notes with a few words about propagators for the non-scalar relativistic fields
--- the vector fields, the tensor fields, the spinor fields, \etc, \etc{}
For all such fields, the Feynman propagator is the time-ordered correlation function of two free
fields in the vacuum state, for example
$$
G_F^{\mu\nu}(x-y)\ =\ \bra0 \bt^* \hat A^\mu(x)\times\hat A^\nu(y) \ket0
\eqno\eq
$$
for the massive vector fields (see \href{\class/hw05.pdf}{\blue homework set\#5} for details), or
$$
S_F^{\alpha\beta}(x-y)\ =\ \bra0 \bt \psib^\alpha(x)\times\hat\Psi^\beta(y) \ket0
\eqno\eq
$$
for the Dirac spinor field $\hat\Psi^\beta(x)$ and its conjugate $\psib^\alpha(x)$
(to be explained in future classes).

All such propagators are Green's functions of the equations of motion for the appropriate fields.
For example, the free massive vector fields obey
$$
\Bigl( g_{\mu\nu}(\partial^2+m^2)\,-\,\partial_\mu\partial_\nu\Bigr) A^\nu\ =\ 0,
\eqno\eq
$$
so the propagator is a Green's function of the differential operator here,
$$
\Bigl( g_{\mu\nu}(\partial^2+m^2)\,-\,\partial_\mu\partial_\nu\Bigr) G_F^{\nu\lambda}(x-y)\
=\ -i\delta_\mu^\lambda\times\delta^{(4)}(x-y).
\eqno\eq
$$
(The proof is part of \href{\class/hw05.pdf}{\blue homework set\#5}.)
Likewise, the free Dirac spinor fields $\Psi^\alpha(x)$ obey the Dirac equation
$$
\bigl(i\gamma^\mu\partial_\mu-m\bigr)_{\alpha\beta}\Psi^\beta(x)\ =\ 0,
\eqno\eq
$$
so the Dirac propagator is a Green's function of the Dirac equation,
$$
\bigl(i\gamma^\mu\partial_\mu-m\bigr)_{\alpha\beta} S_F^{\beta\delta}(x-y)\
=\ -i\delta_\alpha^\delta\times\delta^{(4)}(x-y).
\eqno\eq
$$
(I shall prove this in class in a few weeks.)

Moreover, all such Green's functions involve momentum integrals over poles  along both
mass shells $k_0=\pm\omega_\bk$, and those poles must be regularized.
But for all the Feynman propagators, the poles must be regularized just as we did for the scalar field,
the pole at $k_0=+\omega_\bk$ shifts below the real axis to $+\omega_\bk-i\epsilon$
while the pole at $k_0=-\omega_\bk$ shifts above the real axis to $-\omega_\bk+i\epsilon$.
Consequently, all the Feynman propagators have momentum-space form of
$$
({\rm propagator})^{\rm indices}(x-y)\
=\int\!{d^4k\over(2\pi)^4}\,{i\,e^{-ik(x-y)}\over k^2-m^2+i\epsilon}\times F^{\rm indices}(k)
\eqno\eq
$$
for some simple --- and hopefully non-singular --- function $F^{\rm indices}(k)$.
For example, for the massive vector field
$$
G_F^{\mu\nu}(x-y)\
=\int\!{d^4k\over(2\pi)^4}\,{i\,e^{-ik(x-y)}\over k^2-m^2+i\epsilon}\times
\left( -g^{\mu\nu}\,-\,{k^\mu k^\nu\over m^2}\right),
\eqno\eq
$$
while for the Dirac spinor field
$$
S_F^{\alpha\beta}(x-y)\
=\int\!{d^4k\over(2\pi)^4}\,{i\,e^{-ik(x-y)}\over k^2-m^2+i\epsilon}\times
\bigl(k^\mu\gamma_\mu+m\bigr)^{\alpha\beta}.
\eqno\eq
$$
In general, for a massive field the function $F^{\rm indices}(k)$ is simply a polynomial of $k$
of degree $2\times\rm Spin$.
%
For a massless $\rm spin=\half$ field $F^{\rm indices}(k)$ is also a polynomial,
but for a massless vector field --- or any other kind of a gauge field --- it becomes non-polynomial
and gauge-dependent.
I shall explain the Feynman propagator for the EM fields later in class, probably sometimes in November.

\bye
\chapter{Detalles generales}

%%%%%%%%%%%%%%%%%%%%%%%%%%%%%%%%%%%%%%%%%%%%%%%%%%%%%%%%%%%%%%%%%%%%%%%%%%%%%%%%%%%%%%%%%%%%%%%%%%%%%%%%%%%%%
\section{Four-components}
Recordar que para un espinor de Dirac como el electrón, $e$
\begin{align}
\Psi=\begin{pmatrix}
e_L\\
e_R\\
\end{pmatrix}.
\end{align}
Entonces
\begin{align}
\Psi_L=&P_L\begin{pmatrix}
e_L\\
e_R\\
\end{pmatrix}=\begin{pmatrix}
e_L\\
0\\
\end{pmatrix},&
\Psi_R=&P_R\begin{pmatrix}
e_L\\
e_R\\
\end{pmatrix}=\begin{pmatrix}
0\\
e_R\\
\end{pmatrix},
\end{align}
o nosotros definimos el fermión de Dirac izquierdo como 
Para un fermión de Weyl izquierdO, $l_{\alpha}$, nosotros definimos el fermión de Dirac izquierdo como 

\begin{align}
\label{eq:fc1}
 P_L L=\begin{pmatrix}
l\\
0
\end{pmatrix}.
\end{align}
Nosotros consideramos un fermión de Majorana de 4-componentes definido de un anti-fermión de Weyl izquierdo $\eta^{\alpha}$

\begin{align}
\Psi_M
=\begin{pmatrix}
\eta\\
\eta^\dagger
\end{pmatrix}.
\end{align}
En efecto, nosotros tenemos el correspondiente fermión de Majorana derecho
\begin{align}
\widetilde{N}_R=&P_R \Psi_M=P_R\begin{pmatrix}
\eta\\
\eta^{\dagger}\\
\end{pmatrix}=
\begin{pmatrix}
\eta\\
0
\end{pmatrix}
\end{align}
El correspondiente anti-fermión de Majorana izquierdo es
\begin{align}
\label{eq:fc2}
\overline{\widetilde{N}_R}=&(\widetilde{N}_R)^\dagger\gamma^0=\Psi_M^\dagger P_R \gamma^0
=\Psi_M^\dagger  \gamma^0 P_L \nonumber\\
=&\overline{\Psi_M}P_L \nonumber\\
=&\left(\Psi_M\right)^\dagger \gamma^0 P_L\nonumber\\
=&\begin{pmatrix}
\eta^\dagger & \eta
\end{pmatrix}\begin{pmatrix}
0 & 1\\
1& 0\\
\end{pmatrix}P_L\nonumber\\
=&\begin{pmatrix}
\eta & \eta^\dagger
\end{pmatrix}P_L\nonumber\\
=&\begin{pmatrix}
\eta & 0
\end{pmatrix},
\end{align}
y entonces, para un fermión de Majorana, la partícula es la misma que la antipartícula 
\section{From $2\to 4$}
usando \eqref{eq:fc1} and \eqref{eq:fc1}, nosotros tenemos 
\begin{align}
-\mathcal{L}=&\lambda_{ij}\epsilon_{ab} l^a_i H^b \eta_j+\text{h.c}\nonumber\\
=&\lambda_{ij}\epsilon_{ab}\eta_j l^a_i H^b +\text{h.c}\nonumber\\
=&\lambda_{ij}\epsilon_{ab}\begin{pmatrix}
\eta_j & 0\\
\end{pmatrix} \begin{pmatrix}
l^a_i\\
0
\end{pmatrix} H^b +\text{h.c}\nonumber\\
=&\lambda_{ij}\epsilon_{ab} \overline{\widetilde{N}_R}P_LL^a_i H^b +\text{h.c}\,.
\end{align}
\section{Formulas útiles}
Un numero complejo $Z_{1}$ tiene la siguiente forma:
\begin{align}
Z_{1}=r_{1}e^{i\theta_{1}}
\end{align}
y su conjugado
\begin{align}
Z_{1}^{*}=r_{1}e^{-i\theta_{1}}
\end{align}
De igual forma, un numero complejo $Z_{2}$
\begin{align}
Z_{2}=r_{1}e^{i\theta_{2}}
\end{align}
y su conjugado
\begin{align}
Z_{2}^{*}=r_{1}e^{i\theta_{2}}
\end{align}
\begin{align}
|Z_{1}|^2|Z_{2}|^2&=r_{1}r_{2}
&=Re(Z_{1}Z_{2})
\end{align}
De igual forma, es posible definir 
\begin{align}
Z_{1}=a_{1}+ib_{1}\nonumber\\
Z_{2}=c_{2}+id_{2}
\end{align}
luego
\begin{align}
Z_{1}Z_{2}=a_{1}c_{2}+ia_{1}d_{2}+ib_{1}c_{2}-b_{1}d_{2}
\end{align}
Por tanto
\begin{align}
Im(Z_{1}Z_{2})=(a_{1}d_{2}-b_{1}d_{2})
\end{align}
De igual forma
\begin{align}
Im(Z_{1}Z_{2}^{*})=(-a_{1}d_{2}+b_{1}d_{2})
\end{align}
lo cual conduce
\begin{align}
Im(Z_{1}Z_{2})=-Im(Z_{1}Z_{2}^{*})
\end{align}
\subsection{Parametrizacion de Feymann}
El denominador de la integral esta compuesto por propagadores, para resolver este tipo de integral se introduce parametros de Feymann. Por ejemplo, si se tienen dos propagadores.
\begin{align}
\frac{1}{AB}=\int_0^1 dx\frac{1}{[xA+(1-x)B]^2}
\end{align}
Para resolver la integral
\begin{align}
N&=k^2+(\sigma_{\mu}q^{\mu}+M_{k})\sigma_{\nu}k^{\nu}\nonumber\\
l&=k+px+qy\nonumber\\
\Delta&=-M_{i}^2x(1-x)+(M_{i}^2x+M_{k}^{2})y
\end{align}
luego, la integral adquiere la siguiente forma
\begin{align}
I=2\mu\int_0^1dx\int_0^{1-x}dx\int\frac{d^dl}{(2\pi)^d}\frac{N}{(l^2-\Delta)^3}
\end{align}
teniendo en cuenta
\begin{align}
N&=[l-(px+qy)]^2+(\sigma_{\mu}q^{\mu}+M_{k})\sigma_{\nu}[l^{\nu}-(p^{\nu}x+q^{\nu}y)]\nonumber\\
&=l^2-2l\cdot(px+qy)+(px+qy)^2+(\sigma_{\mu}q^{\mu}+M_{k})\sigma_{\nu}l^{\nu}-(\sigma_{\mu}q^{\mu}+M_{k})\sigma_{\nu}(p^{\nu}x+q^{\nu}y)
\end{align}
el segundo y el cuarto termino se anulan
\begin{align}
I=2\mu\int_0^1dx\int_0^{1-x}dx\int\frac{d^dl}{(2\pi)^d}\frac{l^2+(px+qy)^2-(\sigma_{\mu}q^{\mu}+M_{k})\sigma_{\nu}(p^{\nu}x+q^{\nu}y)}{(l^2-\Delta)^3}
\end{align}
usando las expresiones generales
\begin{align}
\int\frac{d^dl}{(2\pi)^d}\frac{1}{(l^2-\Delta)^n}=\frac{i(-1)^n}{(4\pi)^{\frac{d}{2}}}i\frac{\Gamma(n-\frac{d}{2})}{\Gamma(n)}\left(\frac{1}{\Delta}\right)^{n-\frac{d}{2}}\nonumber\\
\int\frac{d^dl}{(2\pi)^d}\frac{l^2}{(l^2-\Delta)^n}=\frac{i(-1)^{n-1}}{(4\pi)^{\frac{d}{2}}}d\frac{\Gamma(n-\frac{d}{2}-1)}{\Gamma(n)}\left(\frac{1}{\Delta}\right)^{n-\frac{d}{2}-1}
\end{align}
se obtiene
\begin{align}
\int\frac{d^dl}{(2\pi)^d}\frac{1}{(l^2-\Delta)^3}=\frac{-i}{2(4\pi)^2}\frac{1}{\Delta}
\end{align}
\begin{align}
\mu^{4-d}\int\frac{d^dl}{(2\pi)^d}\frac{l^2}{(l^2-\Delta)^3}=\frac{i(-1)^{n-1}}{(4\pi)^{\frac{d}{2}}}d\frac{\Gamma(2-\frac{d}{2})}{\Gamma(n)}\left(\frac{\mu^2}{\Delta}\right)^{2-\frac{d}{2}}
\end{align}
usando
\begin{align}
\frac{\Gamma(2-\frac{d}{2})}{(4\pi)^{\frac{d}{2}}}\left(\frac{1}{\Delta}\right)^{2-\frac{d}{2}}=\frac{1}{(4\pi)^2}\left(\frac{2}{4-d}-ln\Delta-\gamma+ln(4\pi)\right)
\end{align}
\begin{align}
\mu^{4-d}\int\frac{d^dl}{(2\pi)^d}\frac{l^2}{(l^2-\Delta)^3}=\frac{1}{(4\pi)^2}\left(\frac{2}{4-d}-ln\Delta-\gamma+ln(4\pi)+\frac{1}{2}\right)
\end{align}
por lo cual, se obtiene
\begin{align}
a_{0}&=\frac{2i}{(4\pi)^2}\int_{0}^{1} dx \int_{0}^{1-x} dy \left(\Delta_{e}-\frac{1}{2}-ln\frac{\Delta}{\mu^2}-\frac{1}{2\Delta}(px+qy)^2)\right)\nonumber\\
a^{\mu}&=\frac{i}{(4\pi)^2}\int_{0}^{1} dx \int_{0}^{1-x} dy \frac{(p^{\mu}x+q^{\mu}y)}{\Delta}M_{k}\nonumber\\
a^{\mu\nu}&=\frac{i}{(4\pi)^2}\int_{0}^{1} dx \int_{0}^{1-x} dy \frac{q^{\mu}p^{\nu}}{\Delta}
\end{align}
con 
\begin{align}
\Delta_{e}=\left(\frac{2}{4-d}-\gamma+ln(4\pi)\right)
\end{align}
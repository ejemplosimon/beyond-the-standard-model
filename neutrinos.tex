\chapter{Neutrinos}

\section{Weinberg operator}
\label{sec:4weinberg-operator}

Based on Akhmedov, hep-ph/0001264; Langacker, arXiv:1112.5992

With the definition
\begin{align*}
  \psi^c=&C\bar{\psi}^T &\psi^c_L=&C\overline{\psi_L}^T\,.
\end{align*}

The Majorana mass term should be  of the form $\overline{\nu^{c}_L}\nu_L$. Since
$\nu_L$ has $I_3=1/2$, the Majorana mass term
has $I_3=1$\footnote{Any $SU(2)$ spinor $\chi$, $\chi^T i\tau_2$ satisfy that
  \begin{itemize}
  \item both $\chi_1^{\dagger}\chi_2$ and $\chi_1^T i\tau \chi_2$ are invariant.
  \item both $\chi_1^{\dagger}\boldsymbol{\tau}\chi_2$ and $\chi_1^T
    i\tau \boldsymbol{\tau}\chi_2$ transform as vectors.
  \end{itemize}
}. With $L=\begin{pmatrix}\nu_L&e_L\end{pmatrix}^T$
\begin{align*}
\overline{L^c}\boldsymbol{\tau}i\tau_2L&\sim \left( 3,-2 \right)\,.
\end{align*}
One would need an isotriplet scalar field $\Delta\sim (3,2)$, which
either elemental or composite. The term
\begin{align*}
  H^T \boldsymbol{\tau} i\tau_2H&\sim \left( 3,-2 \right),
\end{align*}
can play the role of the composite triplet, where $H=\begin{pmatrix}
H^+ & H^0\end{pmatrix}^T$

Can the operator
\begin{align*}
  \frac{f}{M}\left(\overline{L^c}\boldsymbol{\tau}i\tau_2L\right)\cdot
  \left( H^T \boldsymbol{\tau} i\tau_2H\right)
\end{align*}
be generated as an effective operator at some loop level?


Explicitly we have
\begin{align*}
  \overline{L^c}\boldsymbol{\tau}i\tau_2L=
  &\begin{pmatrix}
    \overline{\nu^c_L} & \overline{e^c_L}
  \end{pmatrix}\left[ 
  \begin{pmatrix}
    0 &1\\
    1 & 0\\    
  \end{pmatrix},
  \begin{pmatrix}
    0 &-i\\
    i & 0\\    
  \end{pmatrix},
  \begin{pmatrix}
    1 &0\\
    0 & -1\\    
  \end{pmatrix}
 \right]
 \begin{pmatrix}
   e_L\\
  -\nu_L\\
 \end{pmatrix}\nonumber\\
 =&\begin{pmatrix}
    \overline{\nu^c_L} & \overline{e^c_L}
  \end{pmatrix}\left[ 
  \begin{pmatrix}
    -\nu_L\\
    e_L\\
  \end{pmatrix},
  i\begin{pmatrix}
    \nu_L\\
    e_L\\
  \end{pmatrix},
  \begin{pmatrix}
    e_L\\
    \nu_L\\
  \end{pmatrix}
 \right]\nonumber\\
  =&\left(\overline{e^c_L}e_L-\overline{\nu^c_L}\nu_L,
    i\left(\overline{\nu^c_L}\nu_L+\overline{e^c_L}e_L\right),
   \overline{\nu^c_L}e_L+\overline{e^c_L}\nu_L
   \right)
\end{align*}
Replacing back $\overline{\nu^c_L}\to H^+$, $\overline{e^c_L}\to H^0$, $\nu_L\to H^+$, and $e_L\to H^0$, we have
\begin{align*}
H^T \boldsymbol{\tau} i\tau_2H=&
  \left(H^0H^0-H^+H^+,
    i\left( H^0H^0+H^+H^+ \right),
   H^+H^0+H^+H^+
   \right).
\end{align*}
\begin{align}
\label{eq:details}
&\left(\overline{L^c}\boldsymbol{\tau}i\tau_2L\right)\cdot
  \left( H^T \boldsymbol{\tau} i\tau_2H\right)\nonumber\\
&=2 \left( 
-\overline{e^c_L}e_LH^+H^+-\overline{\nu^c_L}\nu_LH^0H^0
 +\overline{\nu^c_L}e_LH^+H^0
+\overline{e^c_L}\nu_LH^+H^0\right)\nonumber\\
&=-2\left( \overline{\nu^c_L}H^0-\overline{e^c_L}H^+ \right)
\left( H^0\nu_L-H^+e_L \right)\\
&=-2\left[
  \begin{pmatrix}
    \overline{\nu^c_L}& \overline{e^c_L}
  \end{pmatrix}
  \begin{pmatrix}  
    H^0\\
   -H^+    
  \end{pmatrix}
\right]\left[ 
  \begin{pmatrix}
    H^0 & -H^+
  \end{pmatrix}
  \begin{pmatrix}
   \nu_L\\
   e_L\\    
  \end{pmatrix}
 \right]\nonumber
\end{align}


  We have the Weinberg operator
  \begin{align*}
    \mathcal{L}=&  -\frac{f}{2M}\left(\overline{L^c}\boldsymbol{\tau}i\tau_2L\right)\cdot
  \left( H^T \boldsymbol{\tau} i\tau_2H\right)+\text{h.c}\nonumber\\
=&\frac{f}{M}\left( \overline{L^c}\widetilde{H}^{*} \right)
\left( \widetilde{H}^{\dagger}L \right)+\text{h.c}\nonumber\\
=&\frac{f}{M}\left( \overline{\nu^c_L}H^0-\overline{e^c_L}H^+ \right)
\left( H^0\nu_L-H^+e_L \right)+\text{h.c}\nonumber\\
=&\frac{f}{M}\left(\epsilon_{12}\overline{L^c_1}H_2+\epsilon_{21}\overline{L^c_2}H_1  \right)\left( \epsilon_{12}L_1H_2+\epsilon_{21}L_2H_1 \right)++\text{h.c}\nonumber\\
=&\frac{f}{M}\overline{L^c_a}H_cL_bH_d\epsilon_{ac}\epsilon_{bd}+\text{h.c}\,.
\end{align*}

See Weinberg,  PRL43(1979)1566.


%%% Local Variables: 
%%% mode: latex
%%% TeX-master: "beyond"
%%% End:
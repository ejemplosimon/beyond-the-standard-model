\chapter{Tree level neutrino masses}
Here we calculate the tree level neutrino mases for the Type-I seesaw.

\section{One right-handed neutrino}
We assume to have just one right-handed neutrino $N_R$ 
\begin{align}
  \mathcal{L}_{\nu}=y_{i} \left( L \right)^{\dagger}_i H  N_R 
  +\tfrac{1}{2} M_R\, N_R  N_R + \text{h.c}\,.
\end{align}
In addition to the previous Majorana mass contribution $M_R$, after the spontaneous symmetry breaking the neutral states $\nu_{Li}$ and $N_R$  acquire a Dirac mass contribution
\begin{align}
  \mathcal{L}_{\nu}\supset & M_D^{i} \left( \nu_{Li} \right)^{\dagger} N_R +\frac{1}{2} M_R N_R N_R +\text{h.c} \nonumber\\
  =&  \frac{1}{2}M_D^{i} \left( \nu_{Li} \right)^{\dagger} N_R
     +\frac{1}{2}M_D^{i} N_R\left( \nu_{Li} \right)^{\dagger} +\frac{1}{2} M_R N_R N_R +\text{h.c} \nonumber\\
  =&\frac{1}{2}\begin{pmatrix} \boldsymbol{\nu}_{L}^{\operatorname{T}}  & N_R  \end{pmatrix}
 \begin{pmatrix}
   \mathbf{0}_{3 \times 3} &            \boldsymbol{M}_D \\
   \boldsymbol{M}_D^{\operatorname{T}} & M_R \\
 \end{pmatrix}
\begin{pmatrix} \boldsymbol{\nu}_{L}  \\
    N_R  \end{pmatrix}+\text{h.c} \nonumber\\
  \equiv&\frac{1}{2} \boldsymbol{\chi}^{\operatorname{T}} \boldsymbol{M_{\chi}} \boldsymbol{\chi}+\text{h.c}\,,
\end{align}
where the Dirac neutrino mass vector $\boldsymbol{M}_D$ has components
\begin{align}
  M_D^i=\frac{y^i v}{\sqrt{2}}\,,
\end{align}
$\boldsymbol{\nu}_{L}^{\operatorname{T}}=
\begin{pmatrix}\nu_{L1} & \nu_{L2} & \nu_{L3} \end{pmatrix}^{\operatorname{T}}$, $\boldsymbol{\chi}^{\operatorname{T}}=\begin{pmatrix} \boldsymbol{\nu}_{L}  & N_R  \end{pmatrix}^{\operatorname{T}}$, and
\begin{align}
  \boldsymbol{M_{\chi}}=& \begin{pmatrix}
   \mathbf{0}_{3 \times 3} &            \boldsymbol{M}_D \\
   \boldsymbol{M}_D^{\operatorname{T}} & M_R \\
 \end{pmatrix}
\end{align}
which has eigenvalues
\begin{align}
{  \boldsymbol{M_{\chi}}}_\mp=&\frac{1}{2} \left[ M_R \mp \left( M_R^2 + 4   \boldsymbol{M}_D   \boldsymbol{M}_D^{\operatorname{T}}  \right)^{1/2} \right] \nonumber\\
  =&\frac{1}{2} M_R\left[ 1 \mp  \left( 1 + 4   \boldsymbol{M}_D M_R^{-2}   \boldsymbol{M}_D ^{\operatorname{T}}  \right)^{1/2} \right].
\end{align}
When $M_R^2 \gg \boldsymbol{M}_D  \boldsymbol{M}_D ^{\operatorname{T}} $
\begin{align}
  { \boldsymbol{M_{\chi}}}_\mp\approx &\frac{1}{2} M_R\left[ 1 \mp  \left( 1 +    \boldsymbol{M}_D M_R^{-2}   \boldsymbol{M}_D^{\operatorname{T}}  \right)
   \right]
\end{align}
Therefore
\begin{align}
 M_{\chi^-} \approx&-   \boldsymbol{M}_D M_R^{-1}   \boldsymbol{M}_D^{\operatorname{T}}  \nonumber\\
 M_{\chi^+} \approx&    M_R\,.
\end{align}

The effective neutrino mass matrix, $M_{\nu}$,  in this case   is just a $1\times 1$ matrix
\begin{align}
  M_{\nu}\approx- \boldsymbol{M}_D M_R^{-1} \boldsymbol{M}_D^{\operatorname{T}}\,.
\end{align}



\section{One heavy Dirac fermion}
We need to assume the existence of a heavy Dirac fermion, $F$, and one real scalar singlet $S$
\begin{align}
  \mathcal{L}_{\nu}=y_i \left( L \right)^{\dagger}_i H  F_R 
  + M_F\, \left( F_L \right)^{\dagger}  F_R + h_i \left( F_L \right)^{\dagger} S \nu_{Ri}+   \text{h.c}\,.
\end{align}
In addition to the previous Dirac mass contribution $M_F$, after the full spontaneous symmetry  breaking the neutral states $\nu_{Li}$ and $\nu_{Ri}$  acquire a Dirac mass contributions $M_{Hi}$ and $M_{Si}$ respectively
\begin{align}
  \mathcal{L}_{\nu}\supset & M_{Di} \left( \nu_{Li} \right)^{\dagger}_i  F_R 
  + M_F\, \left( F_L \right)^{\dagger}  F_R + M_{Si} \left( F_L \right)^{\dagger}  \nu_{Ri}+   \text{h.c}\,. \nonumber\\
  =&\begin{pmatrix} \left( \boldsymbol{\nu}_{L} \right)^\dagger  & F_R  \end{pmatrix}
 \begin{pmatrix}
   \mathbf{0}_{3 \times 3} &            \boldsymbol{M}_D \\
   \boldsymbol{M}_S^{\operatorname{T}} & M_F \\
 \end{pmatrix}
\begin{pmatrix} \boldsymbol{\nu}_{R}  \\
    \left( F_L \right)^{\dagger}  \end{pmatrix}+\text{h.c} \nonumber\\
  \equiv& \left( \boldsymbol{\chi}_L \right)^\dagger \boldsymbol{M_{\chi}} \boldsymbol{\chi}_R+\text{h.c}\,,
\end{align}
where the Dirac neutrino mass vectors $\boldsymbol{M}_{D,S}$ have components
\begin{align}
  M_D^i=&\frac{y^i v}{\sqrt{2}}\,,&   M_S^i=&\frac{h^i \langle S\rangle}{\sqrt{2}}\,,
\end{align}
 $\boldsymbol{\nu}_{LR}^{\operatorname{T}}=
\begin{pmatrix}\nu_{LR1} & \nu_{LR2} & \nu_{LR3} \end{pmatrix}^{\operatorname{T}}$, $\boldsymbol{\chi}_{LR}^{\operatorname{T}}=\begin{pmatrix} \boldsymbol{\nu}_{LR}  & \left( F_{RL} \right)^{\dagger}  \end{pmatrix}^{\operatorname{T}}$, and
\begin{align}
  \boldsymbol{M_{\chi}}=& \begin{pmatrix}
   \mathbf{0}_{3 \times 3} &            \boldsymbol{M}_D \\
   \boldsymbol{M}_S^{\operatorname{T}} & M_F \\
 \end{pmatrix}
\end{align}
which has eigenvalues
\begin{align}
{  \boldsymbol{M_{\chi}}}_\mp=&\frac{1}{2} \left[ M_F \mp \left( M_F^2 + 4   \boldsymbol{M}_D   \boldsymbol{M}_S^{\operatorname{T}}  \right)^{1/2} \right].
\end{align}
When $M_F^2 \gg \boldsymbol{M}_D  \boldsymbol{M}_S ^{\operatorname{T}} $ we have effective neutrino mass matrix
\begin{align}
 M_\nu \approx&-   \boldsymbol{M}_D M_F^{-1}   \boldsymbol{M}_S^{\operatorname{T}}  \,,
\end{align}
with $ M_{\chi^+} \approx M_F$.

Note that the recipe to pass from Majorana to Dirac is just to replace $M_D^{\operatorname{T}}\to M_S^{\operatorname{T}} $ 

%%% Local Variables: 
%%% mode: latex
%%% TeX-master: "beyond"
%%% End: